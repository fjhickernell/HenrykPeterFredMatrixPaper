%%%%%%%%%%%%%%%%%%%%%%%%%%%%%%%%%%%%%%%%%%%%%%%%%%%
\documentclass[graybox]{svmult}
% choose options for [] as required from the list  in the Reference Guide
\usepackage{mathptmx}       % selects Times Roman as basic font
\usepackage{helvet}         % selects Helvetica as sans-serif font
\usepackage{courier}        % selects Courier as typewriter font
%\usepackage{type1cm}        % activate if the above 3 fonts are not available on your system
\usepackage{makeidx}         % allows index generation
\usepackage{graphicx}        % standard LaTeX graphics tool
\usepackage{multicol}        % used for the two-column index
\usepackage[bottom]{footmisc}% places footnotes at page bottom
\usepackage{amsmath,amssymb,amsfonts}        % AMS math packages
\usepackage[dvipsnames]{xcolor}
\usepackage{xspace}
% see the list of further useful packages in the Reference Guide
\makeindex             % used for the subject index
% please use the style svind.ist with your makeindex program


%%%%%%%%%%%%%%%%%%%%%%%%%%%%%%%%%%%%%%%%%%%%%%%%%%%%%%%%%%%%%%%%%%%%

\allowdisplaybreaks

\usepackage{booktabs}

\DeclareMathOperator*{\argmax}{argmax}  


        

\newcommand{\tlambda}{\widetilde{\lambda}}
\newcommand{\naturals}{\mathbb{N}}
\newcommand{\me}{\mathrm{e}}
\newcommand{\SEWT}{\sigma_{\;\textup{EWT}}}
\newcommand{\ALG}{\textup{ALG}\xspace}
\newcommand{\EXP}{\textup{EXP}\xspace}
\newcommand{\NOR}{\textup{NOR}\xspace}
\newcommand{\ABS}{\textup{ABS}\xspace}
\newcommand{\SPT}{\textup{SPT}\xspace}
\newcommand{\PT}{\textup{PT}\xspace}
\newcommand{\QPT}{\textup{QPT}\xspace}
\newcommand{\WT}{\textup{WT}\xspace}
\newcommand{\UWT}{\textup{UWT}\xspace}
\newcommand{\calG}{{\mathcal{G}}}
\newcommand{\calH}{{\mathcal{H}}}
\newcommand{\calS}{{\mathcal{S}}}
\newcommand{\C}{{\mathbb{C}}} % complex numbers
\newcommand{\F}{{\mathbb{F}}} % field, finite field
\newcommand{\N}{{\mathbb{N}}} % natural numbers {1, 2, ...}
\newcommand{\Q}{{\mathbb{Q}}} % rationals
\newcommand{\R}{{\mathbb{R}}} % reals
\newcommand{\Z}{{\mathbb{Z}}} % integers

\newcommand{\fh}[1]{{\color{green}{#1}}}
\newcommand{\pk}[1]{{\color{blue}{#1}}}
\newcommand{\hw}[1]{{\color{purple}{#1}}}


%%%%%%%%%%%%%%%%%%%%%%%%%%%%%%%%%%%%%%%%%%%%%%%%%%%%%%%%%%%%%%%%%%%%
\begin{document}
\title*{Exponential tractability of linear tensor product problems: 
finitely many positive eigenvalues}
\titlerunning{Exponential tractability of linear tensor product problems}

% Use \titlerunning{Short Title} for an abbreviated version of
% your contribution title if the original one is too long
\author{Fred J. Hickernell, Peter Kritzer, Henryk Wo\'{z}niakowski}
% Use \authorrunning{Short Title} for an abbreviated version of
% your contribution title if the original one is too long
\institute{Fred J. Hickernell \at Center for Interdisciplinary Scientific Computation,\\ Illinois Institute of Technology, \\
           Pritzker Science Center (LS) 106A, 3105 Dearborn Street, Chicago, IL 60616, USA\\ 
           \email{hickernell@iit.edu}
\and Peter Kritzer \at Johann Radon Institute for Computational and Applied Mathematics (RICAM),\\ Austrian Academy of Sciences\\
     Altenbergerstr. 69, 4040 Linz, Austria\\ \email{peter.kritzer@oeaw.ac.at}
\and Henryk Wo\'{z}niakowski \at Department of Computer Science, Columbia University,\\ New York, NY, USA\\
Institute of Applied Mathematics, University of Warsaw,\\
ul. Banacha 2, 02-097 Warszawa, Poland\\ \email{henryk@cs.columbia.edu}}
%
% Use the package "url.sty" to avoid
% problems with special characters
% used in your e-mail or web address
%
\maketitle

\abstract{In this article we consider the approximation of compact   
linear operators defined over tensor product Hilbert   
spaces. We identify conditions on the singular values of the problem 
under which we can or cannot achieve different notions of exponential tractability.  In particular, we deal 
with the \fh{non-trivial} case where only finitely many of these singular values are different from zero. 
}

\section{Introduction and Preliminaries}
\label{sec:intro}

Tractability of multivariate problems is the subject of a considerable number 
of articles and monographs in the field of Information-Based Complexity.  For an introduction, we refer to the book \cite{TWW88}.  For a recent 
overview of the state of the art in tractability studies we refer to the 
trilogy \cite{NW08}--\cite{NW12}. In this article we study tractability in   
the worst case setting for linear tensor product problems and 
for algorithms that use finitely many   
arbitrary continuous linear functionals.    

The information complexity of a compact linear operator   
$S_d:\calH_d \to \calG_d$ is defined as the minimal number, $n(\varepsilon,S_d)$, of such   
linear functionals needed to find an $\varepsilon$-approximation. 
It is natural to ask how the information complexity of a given problem 
depends on \fh{both} $d$ and $\varepsilon^{-1}$. \pk{(PETER: SLIGHTLY REFORMULATED BELOW
BECAUSE $(s,t)$-WT IMPLIES WE ALLOW EXPONENTIAL DEPENDENCE ON $d$ AND $\varepsilon^{-1}$ TO SOME EXTENT)}

%A problem is called \emph{tractable} if  the information complexity increases with $d$ at a rate that is slower than exponential.  

\pk{In most of the literature on this subject}, 
different notions of tractability are defined in terms of a relationship between $n(\varepsilon,S_d)$ and some function 
$T(d,\max(1,\varepsilon^{-1}))$.  This is called algebraic (ALG) tractability.  
Different notions of \pk{tractability} and the corresponding $T$ are defined \pk{below}. 
For a complete overview of a wide range results 
on algebraic tractability, see \cite{NW08}--\cite{NW12}. 

On the other hand, a relatively recent stream replaces the arguments of $T$ by $(d,1+\log\,\max(1,\varepsilon^{-1}))$. \fh{Would there be an advantage to changing this to $(d,\log\,\max(\me,\varepsilon^{-1}))$ so that we do not have to have $1+ \cdots$?} This situation is 
referred to as exponential (\EXP) tractability, which is the subject of this article. 
Now the complexity of the problem increases only logarithmically as the error tolerance vanishes.  
Again \pk{we} refer to the definitions below.

General compact linear multivariate problems have been studied 
in the recent article \cite{KW18}. Here, we deal with the case of    
tensor product problems for which the singular values of a $d$-variate problem 
are given as products of the singular values of univariate problems. 

\bigskip

Consider two Hilbert spaces $\calH_1$ and $\calG_1$ and a 
compact linear solution operator,
$
S_1: \calH_1 \to \calG_1.
$
Let $\N$ denote the set of positive integers,   
and $\N_0 =\N\cup\{0\}$.  For $d\in \N$, let 
$$
\calH_d\,=\,\calH_1\otimes\calH_1\pk{\otimes}\cdots\otimes\calH_1\quad\mbox{and}\quad
\calG_d\,=\,\calG_1\otimes\calG_1\pk{\otimes}\cdots\otimes\calG_1
$$
be the $d$-\pk{fold} tensor products of the spaces $\calH_1$ and $\calG_1$, respectively. 
Furthermore, let $S_d$ be the linear tensor product operator,
$$ 
S_d=S_1\otimes S_1\otimes \cdots \otimes S_1,
$$
on $\calH_d$. 
In this way, obtain a sequence of compact    
linear solution operators    
$$   
\calS\,=\, \{S_d: \calH_d \to \calG_d\}_{d \in \N}.     
$$   
We now consider the problem of 
approximating $\{S_d(f)\}$ for $f$ from the   
unit ball of $\calH_d$ by means of algorithms   
$\{A_{d,n}: \calH_d \to \calG_d\}_{d \in \N,n\in\N_0}$.    
For $n=0$, we set $A_{d,0}:=0$, and for $n\ge 1$, $A_{d,n}(f)$    
depends on $n$ continuous    
linear functionals $L_1(f), L_2(f),\ldots, L_n(f)$, so that   
\begin{equation}\label{eq:algorithm} 
A_{d,n}(f)=\phi_n(L_1(f),L_2(f),\dots,L_n(f))   
\end{equation}   
for some $\phi_n:\C^n\to \calG_d$ \pk{or $\phi_n:\R^n\to \calG_d$} 
and $L_j\in \calH_d^*$.
We allow an adaptive choice of $L_1,L_2,\ldots,L_n$ as well as $n$, i.e.,   
$L_j=L_j(\cdot;L_1(f),L_2(f),\dots,L_{j-1}(f))$ and $n$ can be a   
function of the $L_j(f)$,    
see \cite{TWW88} and \cite{NW08} for details.
The error of a given algorithm $A_{d,n}$ is measured in the worst case setting, 
which means that we need to deal with  
$$   
e(A_{d,n})\,=\,\sup_{\substack{f\in\calH_d \\ \|f\|_{\calH_d}\le1}}   
\|S_d(f)-A_{d,n}(f)\|_{\calG_d}.   
$$   
However, to assess the difficulty of the approximation problem, we would not 
only like to study the worst case errors of particular algorithms, but 
consider a more general error measure. To this end, let   
$$   
e(n,S_d)\,=\,\inf_{A_{d,n}}\,e(A_{d,n})   
$$   
denote the $n$th minimal worst case error,    
where the infimum is extended over all admissible algorithms $A_{d,n}$ \pk{of the form \eqref{eq:algorithm}}. Then    
the information complexity $n(\varepsilon,S_d)$ is   
the minimal number $n$ of continuous linear functionals    
needed to find an algorithm $A_{d,n}$ that approximates   
$S_d$ with error at most $\varepsilon$. More precisely,    
we consider the absolute (\ABS) and normalized (\NOR) error criteria   
in which    
\begin{align*}   
 n(\varepsilon,S_d)&=n_{\ABS}(\varepsilon, S_d)\, =   
\min\{n\,\colon\,e(n,S_d)\le \varepsilon\},\\   
 n(\varepsilon,S_d)&=n_{\NOR}(\varepsilon,S_d)=   
\min\{n\,\colon\,e(n,S_d) \le \varepsilon\,\|S_d\|\}.   
\end{align*}   

It is known from \cite{TWW88} (see also \cite{NW08})   
that the information complexity is fully determined by    
the singular values of $S_d$, which    
are the same as the square roots of the eigenvalues of    
the compact self-adjoint and positive semi-definite   
linear   
operator $W_d=S_d^\ast S_d:\calH_d\rightarrow\calH_d$.    
We denote these eigenvalues by $\lambda_{d,1}, \lambda_{d,2},\ldots$. Then it is known that 
the information complexity can be expressed in terms of the eigenvalues $\lambda_{d,j}$. Indeed, 
\begin{align}   
n_{\ABS}(\varepsilon, S_d)   
&=\min\{n\,\colon\,\lambda_{d,n+1}\le   
\varepsilon^2\},
\label{eq:infcomplabs}\\   
n_{\NOR}(\varepsilon, S_d)&=   
\min\{n\,\colon\,\lambda_{d,n+1}\le \varepsilon^2 \lambda_{d,1}\}. 
\label{eq:infcomplnor}     
\end{align}
Clearly,    
$n_{\ABS}(\varepsilon, S_d)=0$ for $\varepsilon\ge   
\sqrt{\lambda_{d,1}}=\|S_d\|$, and    
$n_{\NOR}(\varepsilon, S_d)=0$ for $\varepsilon\ge1$.   
Therefore for $\ABS$ we can restrict ourselves to $\varepsilon\in(0,\|S_d\|)$,   
whereas for $\NOR$ to $\varepsilon\in(0,1)$.   
Since $\|S_d\|$ can be arbitrarily large, to deal simultaneously    
with $\ABS$ and $\NOR$ we consider $\varepsilon\in(0,\infty)$.    
It is known that $n_{\ABS/\NOR}(\varepsilon,S_d)$ is finite for all   
$\varepsilon>0$ iff $S_d$ is compact, which justifies our assumption   
about the compactness of~$S_d$.   

We now recall that the spaces $\calH_d$ and $\calG_d$ are tensor product
spaces. It is known that the eigenvalues of $W_d$ are then given as \pk{products} 
of the eigenvalues of the operator $W_1=S_1^\ast S_1:\calH_1\rightarrow\calH_1$, i.e.,
\begin{equation}\label{eq:productform}
\lambda_{d,j}=\prod_{\ell=1}^d \tlambda_{j_\ell}.
\end{equation}
Without loss of generality, we assume that the $\tlambda_j$ are ordered, i.e., $\tlambda_1\ge \tlambda_2 \ge \cdots$. 
The map $j \in \naturals \mapsto (j_1, \ldots, j_d) \in \naturals^d$ exists but does not have a simple explicit form, which makes the tractability analysis challenging.
   
Equations \eqref{eq:infcomplabs}, \eqref{eq:infcomplnor}, and \eqref{eq:productform} enable us to directly relate the 
eigenvalues \pk{$\tlambda_{j}$} to the information complexity $n_{\ABS/\NOR}(\varepsilon, S_d)$. 
As mentioned above, several notions of algebraic or exponential tractability have been defined in terms of how $n_{\mathsf{y}}(\varepsilon, S_d)$ is related to $T(d,\mathsf{z})$, where 
\begin{equation*}
\mathsf{y} \in \{\ABS, \NOR \}, \quad
\mathsf{z} = \begin{cases} \max(1,\varepsilon^{-1}), & \pk{\mbox{in the case of ALG}}, \\
1+\log\, \max(1,\varepsilon^{-1}), & \pk{\mbox{in the case of EXP}}.\end{cases}
\end{equation*}
These definitions are \pk{as follows.}

\bigskip

\pk{$\calS$ is \ldots 

\begin{itemize}
 \item strongly polynomially tractable (\SPT) if there are $C,p,q\ge 0$ such that
$$
 n_{\texttt{y}}(\varepsilon,S_d) \le C \mathsf{z}^q\quad \forall d \in \naturals, \, \varepsilon \in (0,\infty),
$$

\item polynomially tractable (\PT) if there are $C,q\ge 0$ such that
$$
 n_{\texttt{y}}(\varepsilon,S_d) \le C d^p \mathsf{z}^q\quad \forall d \in \naturals, \, \varepsilon \in (0,\infty),
$$

\item quasi-polynomially tractable (\QPT) if there are $C,p\ge 0$ such that
$$
 n_{\texttt{y}}(\varepsilon,S_d) \le C\,\exp\bigl(p\,(1+\log\,d)(1+\log\, \mathsf{z})\bigr)
 \quad \forall d \in \naturals, \, \varepsilon \in (0,\infty),
$$
\item $(s,t)$-weakly tractable ($(s,t)$-\WT) if
$$
\lim_{d+\varepsilon^{-1}\to\infty}\   
\frac{\log\,\max(1,n_{\mathsf{y}}(\varepsilon,S_d))}{d^{\,t}+\mathsf{z}^{s}}\,=\,0,
$$

\item uniformly weakly tractable (\UWT) if $(s,t)$-\WT holds for all $s,t>0$.

\end{itemize}

We use the prefix \ALG- with the above tractability notions in the case $\mathsf{z}=\max(1,\varepsilon^{-1})$ and 
\EXP- in the case $\mathsf{z}=1+\log\, \max(1,\varepsilon^{-1})$. 
}

A recent article \cite{KW18} provides  necessary and sufficient conditions 
on the eigenvalues $\lambda_{d,j}$ of $W_d$ for the various tractability 
notions above. For the special case of linear tensor product spaces 
considered here, it is natural 
to ask for conditions on the eigenvalues $\tlambda_j$ of $W_1$ such that we obtain 
the different kinds of exponential tractability. For results on algebraic tractability 
for tensor product spaces, see again \cite{NW08}--\cite{NW12} and the articles cited therein. 
The notion of $(s,t)$-\WT was introduced in \cite{SW15}, and \UWT was introduced in \cite{S13}. 
See also \cite{WW17} and \cite{KW18} for results on $(s,t)$-\WT and \UWT in the algebraic sense.


Finding  
necessary and sufficient conditions on the $\tlambda_j$ for the different kinds of exponential 
tractability turns out to be a technically difficult question. Here we give first results for the case 
where only finitely many eigenvalues $\tlambda_j$ are nonzero. The more general case, 
where we allow infinitely many positive eigenvalues will be dealt with in a future article. 
%FH The results for finitely many positive eigenvalues will be used as lower bounds for the general case. 


\section{Results}
\label{sec:results}
 
In this section we show results on tractability conditions in terms of the eigenvalues of the 
operator $W_1$ in the case where the number of positive eigenvalues is finite. 

Note that, if all of the $\tlambda_j$ equal zero, then the operators $S_d$ are all zero. Furthermore, 
if only $\tlambda_1>0$ and $\tlambda_2=\tlambda_3=\cdots =0$ (remember that the $\tlambda_j$ are ordered), 
it can be shown that $n_{\ABS/\NOR}(\varepsilon,S_d)=1$ for all $d\ge 1$. Hence, the problem is 
interesting only if at least two of the $\tlambda_j$ are positive, which \pk{we} assume from now 
on. 

Before we state our main result, we state two technical lemmas and \pk{a} theorem proved elsewhere.  The first lemma is well known.

\begin{lemma} \label{sSumLem}
For any $n \in \N$  and $\pk{a_1,a_2,\ldots,a_n} \ge 0$ we have:
\begin{itemize}
\item For $s\ge1$,
\begin{equation*}
\frac1{n^{s-1}} (a_1 + \cdots + a_n)^s
\le a_1^s + \cdots + a_n^s \le
(a_1 + \cdots + a_n)^s.
\end{equation*}
\item For $s\le1$
\begin{equation*}
\qquad
\qquad
(a_1 + \cdots + a_n)^s
\le a_1^s + \cdots + a_n^s \le n^{s-1}(a_1 + \cdots + a_n)^s.
\end{equation*}
\end{itemize}
\end{lemma}


\begin{lemma} \label{binomBdLem} For all $k, n \in \N$ with $k < n$, it follows that
\begin{equation*}
      \max\left\{\left(\frac nk\right)^k,  \left(\frac n{n-k}\right)^{n-k} \right\}
      \le \binom{n}{k} \le \min\left\{\left(\frac {\me n}k\right)^k,  \left(\frac {\me n}{n-k}\right)^{n-k} \right\}.
\end{equation*}
\end{lemma}
\begin{proof}
It is easy to see that
                \begin{align*}
                \left(\frac nk\right)^k  = \pk{\frac{n}{k}\cdots\frac{n}{k}}           
                & \le \frac{n (n-1) \cdots (n - k+1)}{\pk{k (k-1) \cdots  1}}
                 =\binom{n}{k} = \binom{n}{n-k} \\
                & \le \frac{n^k}{k!} = \left(\frac nk\right)^k \frac{k^k}{k!} \le  \left(\frac {\me n}k\right)^k,
                \end{align*}
and the estimates of the lemma follow.$\hfill\Box$
\end{proof}

\begin{theorem} \cite{KW18} \label{thm:general}   
$\calS$ is \EXP-$(s,t)$-\WT-\ABS/\NOR iff   
\begin{equation}\label{eq:EXP-WT}   
\sup_{d\in\N} \SEWT (d,s,t,c)    
 < \infty\quad \forall c>0,   
\end{equation}   
where   
\[   
 \SEWT (d,s,t,c):=\exp (-cd^{\,t})\,\sum_{j=1}^\infty    
\exp \left(-c \left[   
1+   
\log\left(2\,\max\left(1,\frac{{\rm   
        CRI}_d}{\lambda_{d,j}}\right)\right)   
\right]^s \right),   
\]   
where 
$$   
 \mathrm{CRI}_d=\begin{cases}   
                 1 &\mbox{for $\ABS$},\\   
                 \lambda_{d,1} &\mbox{for $\NOR$}.   
                 \end{cases}   
$$  
\end{theorem}   



We now state and prove our main result in this article.
\begin{theorem}\label{thm:main}
Let
$$
\tlambda_1\ge\cdots\ge\tlambda_m>\tlambda_{m+1}=\cdots=0\ \ \
\mbox{for some}\ \ m\ge2.
$$
\EXP-$(s,t)$-\WT-\ABS holds iff one of the following conditions is true:
\begin{itemize}
\item $t>1$ \ \ \mbox{or}
\item $t\le 1$, $s>1$, $\tlambda_1\le1$, and $\tlambda_2<1$.
\end{itemize}
\EXP-$(s,t)$-\WT-\NOR holds iff one of the following conditions is true:
\begin{itemize}
\item $t>1$ \ \ \mbox{or}
\item $t\le 1$, $s>1$, and $\tlambda_1>\tlambda_2$.
\end{itemize}
Furthermore, \EXP-\UWT, \EXP-\QPT, \EXP-\PT, and \EXP-\SPT do not hold under any conditions.

\end{theorem}

\begin{proof}

We know from Theorem \ref{thm:general} that \EXP-$(s,t)$-\WT-\ABS holds iff

        \begin{equation} \label{EXPWTcond}
        \sup_{d \in \naturals}  \SEWT(d,s,t ,c) < \infty \quad \forall c>0,\\
        \end{equation}
        where
        \begin{align}
        \nonumber
        \lefteqn{\SEWT(d,s,t,c )} \\
        \label{SEWTSingSum}
         & :=  \sum_{j=1}^\infty \exp\left( -c \left \{d^t + \left [ \log(2\me) + 
         \log \left(\max\left(1, \lambda_{d,j}^{-1} \right) \right) \right]^s \right \}\right) \\
         \label{SEWTmultisum}
        &=  \sum_{j_1=1}^m \cdots   \sum_{j_d=1}^m   \exp\left( -c \left \{ d^t + 
        \left [ \log \left(2 \me \max\left(1, \prod_{\ell = 1}^d \tlambda_{j_\ell}^{-1} \right) \right) \right]^s  \right\} \right).
        \end{align}



We consider first a number of cases depending on $t,s,\tlambda_1$ and
        $\tlambda_2$ and verify first when \EXP-$(s,t)$-\WT-\ABS holds or does
        not hold.  The corresponding proofs for \EXP-$(s,t)$-\WT-\NOR then follow routinely. \fh{The first two cases establish exponential tractability, and both arguments require that $m$ be finite.  The next three cases establish the absence of tractability, and do not require that $m$ be finite.}


\paragraph{\textit{Case 1.} $t>1 \implies $ \EXP-$(s,t)$-\WT-\ABS}

        There are only $m^d$ nonzero eigenvalues,
$\lambda_{d,j}$, so it follows that
        \begin{align*}
        \sup_{d \in \naturals}
\SEWT(d,s,t,c )
&\le \sup_{d \in \naturals} m^d \exp\left( -c
d^{\,t}  \right) \\
        & = \sup_{d \in \naturals}  \exp\left(d \log(m) -c
d^{\,t}  \right) < \infty \quad \forall c > 0.
        \end{align*}
        This argument requires $t$ strictly greater than one
for $m\ge2$ because of the requirement of a finite supremum
for all positive $c$.

\paragraph{\textit{Case 2.} $t\le 1, \ s > 1,  \ \tlambda_1 \le 1, \ \& \ \tlambda_2< 1 \implies$ \EXP-$(s,t)$-\WT-\ABS} \label{Case111g} 

There are a total of \pk{$m^d$} terms of the form $\prod_{\ell = 1}^d \tlambda_{j_\ell}^{-1}$ for $j_1, \ldots, j_d \in \{1, \ldots, m\}$ 
in the multi-sum in \eqref{SEWTmultisum}.  For $k = 0, \ldots, d$ there are $(m-1)^k\binom{d}{k}$ 
of these terms containing the factor of $\tlambda_1^{-1}$ exactly $d-k$ times.  
Such terms are bounded above below by $\tlambda_2^{-k}$. Dropping $\log(2\me)$ in \eqref{SEWTmultisum}, we obtain
$$
\SEWT(d,s,t,c ) \le \exp(-cd^{\,t})\,\sum_{k=0}^d\binom{d}{k}
\,(m-1)^k\,\exp\left(-c\,\bigl[\log(\tlambda_{2}^{-k})\bigr]^s\right).
$$
We bound
$\binom{d}{k}$ by $(\me\, d/k)^k$ due to Lemma \ref{binomBdLem}. Hence, we have
\begin{align*}
\SEWT(d,s,t,c) & \le 1 + \exp(-cd^{\,t})\,  d\,
\max_{k=1,\dots,d}\,\exp(f(k)) \\
& \le 1 +  \exp\bigl(-cd^{\,t} + \log(d) + f(k_{\max})\bigr),
\end{align*}
where
\begin{align*}
f(k)&=k+k\log(d/k)+k\log(m-1)-ck^s\bigl[\log(\tlambda_2^{-1})\bigr]^s, \\
f'(k)&= \log(d/k)+ \log(m-1)- c s k^{s-1}\bigl[\log(\tlambda_2^{-1})\bigr] , \\
f(k_{\max})& = \max_{k\in[1,d]}f(k)  \ge \max_{k=1,\dots,d}f(k).
\end{align*}
For $d$ large enough, we have
\begin{align*}
f'(1)&= \log(d) +\log(m-1)- c s(\log(\tlambda_2^{-1}))^s > 0,\\
f'(d)&= \log(m-1)-c s d^{s-1}(\log(\tlambda_2^{-1}))^s < 0,
\end{align*}
hence,
the maximum occurs in the interior.  By setting the $f'(k)=0$, we obtain
\begin{align*}
0  & = \log(d/k_{\rm max}) + \log(m-1)-
csk_{\rm max}^{s-1}(\log(\tlambda_2^{-1}))^s, \\
f(k_{\rm max})&=k_{\rm max}+k_{\rm max}\log(d/k_{\rm max})
+k_{\rm max}\log(m-1)-ck_{\rm
        max}^{s}\left(\log(\tlambda_2^{-1})\right)^s\\
&= k_{\rm max}+c\,(s-1)\,k_{\rm
        max}^{s}\left(\log(\tlambda_2^{-1})\right)^s.
\end{align*}

The nonlinear equation defining $k_{\max}$ above implies that
\begin{align*}
k_{\rm max} &=
\mathcal{O}\left(\left(\log(d)\right)^{1/(s-1)}\right),\\
f(k_{\rm max})&=
\mathcal{O}\left( \left(\log(d)\right)^{s/(s-1)}\right).
\end{align*}
Combining these estimates we conclude that
\begin{multline*}
\sup_{d\in \naturals}\SEWT(d,s,t,c )
\\
\le 1 + \sup_{d\in\naturals}
\exp\left(-cd^{\,t}+\log(d)+ \mathcal{O}\left((\log(d))^{s/(s-1)}\right)   \right)<\infty \
\ \forall\,c>0.
\end{multline*}
Hence, we have \EXP-$(s,t)$-\WT-\ABS in this case.

\paragraph{\textit{Case 3.} $t\le 1 \ \&  \ \tlambda_1 > 1 \implies$ NO \EXP-$(s,t)$-\WT-\ABS} \label{Case111f}

        Choose the smallest non-negative $r$ such that $\tlambda_2 \ge
        \tlambda_1^{-r}$, and for every $d > r+2$, let $k = \lfloor
        d/(r+2) \rfloor$.
        Then it follows that
        \begin{equation*}
        k \le \frac{d}{r+2}, \qquad d - k - kr \ge d\left[1 -\frac{r+1}{r+2} \right] = \frac{d}{r+2}.
        \end{equation*}
        Focusing on just these eigenvalues of the form
        \begin{equation*}
        \lambda_{d,j} = \tlambda_1^{d-k}  \tlambda_2^{k} =  \tlambda_1^{d-k-kr}  \tlambda_1^{kr}\tlambda_2^{k} \ge
        \tlambda_1^{d-k-kr}  \tlambda_1^{kr} \tlambda_1^{-kr} =\, \tlambda_1^{d-k-kr} \ge \tlambda_1^{d/(r+2)} \,\ge\, 1,
        \end{equation*}
        it follows that $\SEWT(d,s,t,c )$ has the following lower
        bound via \eqref{SEWTSingSum} and Lemma \ref{binomBdLem}.
        \begin{align*}
        \nonumber
        \SEWT(d,s,t,c )
        & \ge \binom{d}{k} \exp\left( -c \left \{d^t + \left [ \log(2\me) + \log \left(\max\left(1, \lambda_{d,j}^{-1} \right) \right) \right]^s \right \}\right) \\
        & \ge \left(\frac{d}{k}\right)^k \exp\left( -c \left \{d^t + \left [ \log(2\me) \right]^s \right \}\right) \\
        & \ge \left(r+2\right)^{d/(r+2) - 1} \exp\left( -c \left \{d^t + \left [ \log(2\me) \right]^s \right \}\right) \\
        & = \frac{\bigl[(r+2)^{1/(r+2)}\bigr]^d}{r+2} \exp\left( -c \left \{d^t + \left [ \log(2\me) \right]^s \right \}\right).
        \end{align*}
        Since $(r+2)^{1/(r+2)} > 1$ and $t \le 1$, then $\SEWT(d,s,t,c ) \to \infty$ for small $c$ as $d \to \infty$, regardless of the value of $s$.
        Hence, we do not have \EXP-$(s,t)$-\WT-\ABS.


\paragraph{\textit{Case 4.} $t\le 1 \ \&  \ \tlambda_1\ge\tlambda_2\ge1 \implies$ NO \EXP-$(s,t)$-\WT-\ABS }
 We have $2^d$ eigenvalues no smaller than $1$. Therefore
        $$
        \SEWT(d,s,t,c )\ge 2^d\exp\left(-c(d^t+[\log(2e)]^s)\right)
        \to \infty
        $$
        for small $c$ independently of $s$.
        Hence, we do not have \EXP-$(s,t)$-\WT-\ABS.



\paragraph{\textit{Case 5.} $t\le 1 \ \&  \ s \le 1 \implies$ NO \EXP-$(s,t)$-\WT-\ABS} \label{Case111e} 

We have $2^d$ eigenvalues no smaller than $\tlambda_2^d$.
We then have
\begin{align*}
      \SEWT(d,s,t,c) & \ge 2^d
            \exp\left(-c\left(d^t +
            \left[\log\left(2\me \max(1,\tlambda_2^{-d})
            \right)\right]^s\right)\right)\\
      & = 
      \exp\left(d \log \, 2 - cd^t -c\left[\log (2\me) +d\log \, \max(1,\tlambda_2^{-1})\right]^s\right).
\end{align*}
Since $s,t\le 1$ and since $c$ can be arbitrarily small,
we see that this latter term is not bounded for $d\rightarrow\infty$.
        Hence, we do not have \EXP-$(s,t)$-\WT-\ABS.

{}From the analysis of all these cases, we see that
\EXP-$(s,t)$-\WT-\ABS holds only when $t>1$, and
when $t\le 1$, $s>1$,
$\tlambda_1\le1$, and $\tlambda_2<1$. This completes the proof for \ABS.

%FJH think that I have simplified this proof.
\iffalse
Consider the eigenvalues of the form  $\tlambda_1^k\tlambda_2^{d-k}$
for $k\in\{0,\ldots,d\}$.
We then have
\begin{align*}
      \lefteqn{\SEWT(d,s,t,c) \ge
            \sum_{k=0}^d {d \choose k}
            \exp\left(-c\left(d^t +
            \left[\log\left(2\me \tlambda_1^{-k}\tlambda_2^{-(d-k)}
            \right)\right]^s\right)\right)}\\
      & = \exp(-cd^t)\sum_{k=0}^d {d \choose k}
      \exp\left(-c\left[\log (2\me) +
      k\log\tlambda_1^{-1}+(d-k)\log\tlambda_2^{-1}\right]^s\right).
\end{align*}
Then, using Lemma \ref{binomBdLem} and Lemma \ref{sSumLem}, it follows that
        \begin{align*}
                \lefteqn{\SEWT(d,s,t,c)\ge\exp(-cd^t)}\\
                &\qquad \times\sum_{k=0}^d {d \choose k}\exp\left(-c\left(\left[
                \log (2\me)\right]^s +
                \left[k\log\tlambda_1^{-1}\right]^s+\left[(d-k)\log\tlambda_2^{-1}\right]^s\right)\right)\\
                &\ge \exp(-cd^t)\\
                &\qquad\times {d \choose \lfloor d/2 \rfloor}
                \exp\left(-c\left(\left[\log (2\me)\right]^s +
                \left[\lfloor d/2 \rfloor\log\tlambda_1^{-1}\right]^s+
                \left[\lceil d/2 \rceil\log\tlambda_2^{-1}\right]^s\right)\right)\\
                &\ge \exp(-cd^t)\\
                &\times\left(\frac{d}{\lfloor d/2 \rfloor}
                \right)^{\lfloor d/2 \rfloor}
                \exp\left(-c\left(\left[\log (2\me)\right]^s +
                \left[\lfloor d/2 \rfloor\log\tlambda_1^{-1}\right]^s+
                \left[\lceil d/2 \rceil\log\tlambda_2^{-1}\right]^s\right)\right)\\
                &\ge \exp(-cd^t)\\
                &\times\left(\frac{d}{d/2}\right)^{d/2-1}
                \exp\left(-c\left(\left[\log (2\me)\right]^s +
                \left[\lfloor d/2 \rfloor\log\tlambda_1^{-1}\right]^s+
                \left[\lceil d/2 \rceil\log\tlambda_2^{-1}\right]^s\right)\right)\\
                &= \exp(-cd^t)\\
                &\qquad\times 2^{d/2-1}
                \exp\left(-c\left(\left[\log (2\me)\right]^s +
                \left[\lfloor d/2 \rfloor\log\tlambda_1^{-1}\right]^s+
                \left[\lceil d/2 \rceil\log\tlambda_2^{-1}\right]^s\right)\right)\\
                &=\exp(-cd^t)\, \frac{1}{2}\\
                &\qquad\times
                \exp\left(d\log(\sqrt{2})-c\left(\left[\log (2\me)\right]^s +
                \left[\lfloor d/2 \rfloor\log\tlambda_1^{-1}\right]^s+
                \left[\lceil d/2 \rceil\log\tlambda_2^{-1}\right]^s\right)\right).
        \end{align*}
        Since $s,t\le 1$ and since $c$ can be arbitrarily small,
        we see that the latter term is not bounded for $d\rightarrow\infty$.
        Hence, we do not have \EXP-$(s,t)$-\WT-\ABS in this case.

{}From the analysis of all these cases, we see that
\EXP-$(s,t)$-\WT-\ABS holds only when $t>1$, and
when $t\le 1$, $s>1$,
$\tlambda_1\le1$, and $\tlambda_2<1$. This completes the proof for \ABS.
\fi

\vskip 1pc
We turn to \NOR. This corresponds to considering the ratios
$\lambda_{d,1}/\lambda_{d,j}$ which are at least $1$.
We know that \EXP-$(s,t)$-\WT-\NOR holds iff
$$
\sup_{d \in \naturals}  \SEWT(d,s,t ,c) < \infty \quad \forall c>0,
$$
where
\begin{eqnarray*}
\SEWT(d,s,t,c )&=&
\sum_{j=1}^\infty \exp\left( -c \left \{d^t +
\left [ \log(2\me) + \log
  \left(\frac{\lambda_{d,1}}{\lambda_{d,j}}\right)
\right) \right]^s\right) \\
&=&  \sum_{j_1=1}^\infty \cdots
\sum_{j_d=1}^\infty  \exp\left( -c \left\{ d^t +
\left [ \log \left(2 \me \prod_{\ell = 1}^d
    \frac{\tlambda_1}{\tlambda_{j_\ell}}
 \right) \right]^s \right\}\right).
 \end{eqnarray*}
Hence, it is the same as \ABS if we assume that $\tlambda_1=1$.
Using the results for \ABS with $\tlambda_1=1$ we obtain the results
for \NOR. This completes the proof for \EXP-$(s,t)$-\WT.

\bigskip

Regarding all other tractability notions,
we know from above that we do not have \EXP-$(s,t)$-\WT when $t\le 1$ and $s\le 1$. Since \EXP-$(s,t)$-\WT is a weaker tractability
notion than all other tractability notions considered here, we cannot have any other stronger kind of tractability.

This completes the proof of Theorem \ref{thm:main}.$\hfill\Box$

\end{proof}

\fh{Should we give a brief discussion about how (I think) we can get SPT and stronger notions of tractability in the ALG case, even though we cannot in the EXP case?}


%
\begin{acknowledgement}
The authors thank the MATRIX institute in Creswick, VIC, Australia, and its staff for supporting their 
stay during the program ``On the Frontiers of High-Dimensional Computation'' in June 2018. 

F.~J.~Hickernell gratefully acknowledges support by the United States National Science Foundation grant DMS-1522687.

P.~Kritzer gratefully acknowledges support by the Austrian Science Fund (FWF): Project F5506-N26, which is part
of the Special Research Program ``Quasi-Monte Carlo Methods: Theory and Applications''. 

H.~Wo\'{z}niakowski gratefully acknowledges the support of the National Science Centre, Poland, based on the decision 
DEC-2017/25/B/ST1/00945.
\end{acknowledgement}
%



\begin{thebibliography}{99.}%

\bibitem{KW18} Kritzer, P., Wo\'{z}niakowski, H.: Simple characterizations of exponential tractability   
for linear multivariate problems. Submitted (2018)

\bibitem{NW08} 
Novak, E., Wo\'zniakowski, H.: Tractability    
of Multivariate Problems, Volume I: Linear Information. EMS, Z\"urich (2008)   
   
\bibitem{NW10}   
Novak, E., Wo\'zniakowski, H.: Tractability    
of Multivariate Problems, Volume II: Standard Information    
for Functionals. EMS, Z\"urich (2010)   
   
\bibitem{NW12}   
Novak, E., Wo\'zniakowski, H.: Tractability    
of Multivariate Problems, Volume III:    
Standard Information for Operators. EMS, Z\"urich (2012)   

\bibitem{S13}   
Siedlecki, P.: Uniform weak tractability.   
J. Complexity, 29, 438--453 (2013)   

\bibitem{SW15} Siedlecki, P., Weimar, M.: Notes on $(s,t)$-weak tractability: 
a refined classification of problems with (sub)exponential information complexity.
J. Approx. Theory 200, 227--258 (2015) 

\bibitem{TWW88}   
Traub, J.F., Wasilkowski, G.W., Wo\'zniakowski, H.:   
Information-Based Complexity. Academic Press, New York (1988)   

\bibitem{WW17} Werschulz, A., Wo\'{z}niakowski, H.: A new    
characterization of $(s,t)$-weak tractability. J. Complexity 38, 68--79 (2017) 

\end{thebibliography}
\end{document}
